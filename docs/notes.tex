\documentclass[11pt]{article}

\usepackage{amsmath}
\usepackage{fullpage}
\usepackage{stmaryrd}

\newcommand{\ff}{\mathbf{f}}
\newcommand{\nn}{\mathbf{n}}
\newcommand{\uu}{\mathbf{u}}
\newcommand{\xx}{\mathbf{x}}


\begin{document}
We are interested in describing the dynamics microtubules between the
centrosomes and the nucleus of a cell.  Microtubules are polymers that
can grow and shrink, and they form part of the cytoskeleton (giving the
cell structure) and play a key role in mitosis (cell division).

\subsection*{April 9, 2019}
\begin{itemize}
  \item Skyped with Yuan and Reza

  \item Describe the nucleus as a vesicle (locally inextensible and
  resistance to bending), and the centrosomes as a single point.

  \item Microtubules link the centrosome to the nucleus.

  \item Since microtubules do not bend much (???), they can only bind to
  the nucleus at locations in the field of view.

  \item The probability of linking at a particular location depends
  exponentially on the distance from the centrosome.  Therefore, the
  pulling force due to the microtubules can be modelled with an
  exponential-type potential.

  \item As the microtubules grow, they start to buckle, and it is
  thought that this causes a pushing force which causes the centrosome
  to move.

  \item In experiments, the nucleus can be made softer.  This can be
  done in the code with the bending rigidity parameter.

  \item For our first experiments, we will focus on only the pulling
  forces.

  \item Experiments indicate that very sharp corners can form.

  \item It is unclear if the receptors for the microtubules are
  distributed uniformly on the nucleus, if they congregate (cluster)
  at a certain location, or if they are able to migrate as the nucleus
  deforms.  Two possible models are that the receptors are distributed
  uniformly with respect to arclength, or they are distributed uniformly
  with respect to curvature.
\end{itemize}

\section{Governing Equations}
We represent the nucleus as a two-dimensional inextensible bilayer
structure that is filled with and submerged in a viscous incompressible
fluid (inextensible vesicle).  Then the fluid equations in the fluid
bulk are
\begin{align}
  \mu_e \Delta \uu &= \nabla p, \quad \xx \in \Omega, \\
  \nabla \cdot \uu &= 0, \quad \xx \in \Omega,
\end{align}
and inside the nucleus they are
\begin{align}
  \mu_i \Delta \uu &= \nabla p, \quad \xx \in \omega, \\
  \nabla \cdot \uu &= 0, \quad \xx \in \omega.
\end{align}
The viscosity ratio $\nu = \mu_e/mu_i$ is a parameter in our model.  The
boundary conditions on the nucleus $\gamma$, which is parameterized as
$\xx(s,t)$, are 
\begin{align}
  \dot{\xx} &= \uu \\
  \llbracket T \rrbracket \nn &= \ff
\end{align}
where $\llbracket T \rrbracket$ is the traction jump across the
membrane.  The interfacial force $\ff$ is a combination of bending,
tension, and point forces.  The bending force penalizes curvature and is
given by
\begin{align}
  \ff = -\kappa_b \xx_{ssss},
\end{align}
where $\kappa_b$ is the bending stiffness.  The tension force is
\begin{align}
  (\sigma \xx_s)_s,
\end{align}
where the tension $\sigma$ acts as a Lagrange multiplier to satisfy the
local inextensibility constraint.  We are interested in the new point
forces that come from a microtubule linking the centrosome to the
nucleus.  We assume there are two centrosomes placed symmetrically about
the nucleus

Based on the probability of a microtubule connecting a centrosome to the
nucleus, the point force potential at a point $\xx$ on the nucleus
membrane is
\begin{align}
  \phi(\xx) = W\exp\left(-\frac{\|\xx - \xx_0\|^2}{d} 
                         -\frac{\|\xx + \xx_0\|^2}{d}\right),
\end{align}
where $\pm \xx_0$ are the fixed locations of the centrosomes, $d$ is
a parameter corresponding to a length scale of the potential, and $W$
sets the magnitude of the point forces.  We always assume that the
nucleus is centered at the origin and has symmetry in both the $x$ and
$y$ directions.


\section{Results}










\end{document}
